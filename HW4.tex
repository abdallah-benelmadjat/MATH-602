\documentclass{article}
\usepackage{amsmath}
\usepackage{amsfonts}
\usepackage{geometry}
\usepackage{graphicx}
\usepackage{hyperref}
\usepackage{xcolor}
\geometry{margin=1in}

\date{}

\begin{document}

\begin{titlepage}
    \centering
    \includegraphics[width=0.5\textwidth]{uwm-logo.png} \\
    \vspace{1cm}
    {\large \textbf{Mechanical Engineering Department}}\\[0.5em]
    {\large ME604/704 - Advanced Engineering Mathematics II}\\[0.5em]
    {\large Instructor: Dr. Istvan Lauko}\\[0.5em]
    {\large Spring 2025}\\[2cm]
    
    {\large \textbf{Class:} MATH 602}\\[0.5em]
    {\large \textbf{Name:} Abdallah Benelmadjat}\\[0.5em]
    {\large This homework has been written using \LaTeX}\\
    {\large Find the LaTeX GitHub source code by clicking \href{https://github.com/abdallah-benelmadjat/MATH-602/blob/main/HW4.tex}{\textcolor{blue}{here}}}\\

\end{titlepage}

\newpage

% --- First Problem ---
\section*{Problem 1}
\subsection*{a) Let $L = 2$}

Given:
\[
g(x) =
\begin{cases}
0, & -2 \le x < 0 \\
2, & 0 \le x \le 2
\end{cases}
\quad \text{(period = 4)}
\]

General form:
\[
g(x) = a_0 + \sum_{n=1}^\infty \left[ a_n \cos\left( \frac{n\pi x}{2} \right) + b_n \sin\left( \frac{n\pi x}{2} \right) \right]
\]

\paragraph{Zeroth coefficient:}
\[
a_0 = \frac{1}{2L} \int_{-2}^{2} g(x)\,dx = \frac{1}{4} \left( \int_{-2}^{0} 0\,dx + \int_0^2 2\,dx \right)
\]
\[
= \frac{1}{4} (0 + 2 \cdot (2 - 0)) = \frac{1}{4}(4) = 1
\]

\paragraph{Cosine coeffs:}
\[
a_n = \frac{1}{2} \int_{-2}^{2} g(x) \cos\left( \frac{n\pi x}{2} \right)\,dx
\]
Only second integral survives:
\[
= \frac{1}{2} \int_0^2 2 \cos\left( \frac{n\pi x}{2} \right)\,dx = \int_0^2 \cos\left( \frac{n\pi x}{2} \right)\,dx
\]
\[
= \left. \frac{2}{n\pi} \sin\left( \frac{n\pi x}{2} \right) \right|_0^2 = \frac{2}{n\pi} ( \sin(n\pi) - \sin(0) ) = 0
\]
\[
\Rightarrow a_n = 0
\]

\paragraph{Sine coeffs:}
\[
b_n = \frac{1}{2} \int_{-2}^{2} g(x) \sin\left( \frac{n\pi x}{2} \right)\,dx
= \frac{1}{2} \int_0^2 2 \sin\left( \frac{n\pi x}{2} \right)\,dx
= \int_0^2 \sin\left( \frac{n\pi x}{2} \right)\,dx
\]
\[
= \left. -\frac{2}{n\pi} \cos\left( \frac{n\pi x}{2} \right) \right|_0^2
= -\frac{2}{n\pi} \left( \cos(n\pi) - \cos(0) \right)
\]
\[
= -\frac{2}{n\pi} \left( (-1)^n - 1 \right) = \frac{2}{n\pi} (1 - (-1)^n)
\]
\[
\Rightarrow b_n =
\begin{cases}
0, & \text{if } n \text{ even} \\
\frac{4}{n\pi}, & \text{if } n \text{ odd}
\end{cases}
\]

\paragraph{Final series:}
\[
g(x) = 1 + \sum_{\substack{n=1 \\ n\ \text{odd}}}^\infty \frac{4}{n\pi} \sin\left( \frac{n\pi x}{2} \right)
\]

Truncated:
\[
g(x) \approx 1 + \frac{4}{\pi} \sin\left( \frac{\pi x}{2} \right) + \frac{4}{3\pi} \sin\left( \frac{3\pi x}{2} \right)
\]

\subsection*{b) Let \( L = \frac{1}{2} \)}

Given:
\[
g(x) =
\begin{cases}
\frac{1}{2} + x, & -\frac{1}{2} \le x < 0 \\
\frac{1}{2} - x, & 0 \le x \le \frac{1}{2}
\end{cases}
\quad \text{(even function)}
\]

\paragraph{Zeroth coefficient:}
\[
a_0 = \frac{1}{2L} \int_{-1/2}^{1/2} g(x)\,dx = \int_{-1/2}^{1/2} g(x)\,dx
\]
We break it:
\[
= \int_{-1/2}^{0} \left( \frac{1}{2} + x \right)\,dx + \int_0^{1/2} \left( \frac{1}{2} - x \right)\,dx
\]
We do the first:
\[
\int_{-1/2}^{0} \left( \frac{1}{2} + x \right)\,dx = \left[ \frac{1}{2}x + \frac{x^2}{2} \right]_{-1/2}^{0}
= 0 - \left( -\frac{1}{4} + \frac{1}{8} \right) = \frac{1}{8}
\]
Now the second:
\[
\int_0^{1/2} \left( \frac{1}{2} - x \right)\,dx = \left[ \frac{1}{2}x - \frac{x^2}{2} \right]_0^{1/2}
= \frac{1}{4} - \frac{1}{8} = \frac{1}{8}
\]
Total:
\[
a_0 = \frac{1}{8} + \frac{1}{8} = \frac{1}{4}
\]

\paragraph{Cosine coeffs:}
Even function:
\[
a_n = 2 \int_0^{1/2} \left( \frac{1}{2} - x \right) \cos(2n\pi x)\,dx
\]

We use integration by parts or known result:
\[
a_n = \frac{1}{2n^2\pi^2}
\]

\paragraph{Sine coeffs:}
\[
b_n = 0 \quad \text{(odd × even = odd → integral = 0)}
\]

\paragraph{Final series:}
\[
g(x) = \frac{1}{4} + \sum_{n=1}^\infty \frac{1}{2n^2\pi^2} \cos(2n\pi x)
\]

Truncated:
\[
g(x) \approx \frac{1}{4} + \frac{1}{\pi^2} \left( \frac{1}{1^2} \cos(2\pi x) + \frac{1}{3^2} \cos(6\pi x) \right)
\]

\newpage
\section*{Problem 2}

Let:
\[
g(x) = \frac{x^2}{2}, \qquad L = \pi, \qquad \text{function is } 2\pi\text{-periodic}
\]

The general Fourier series for a function \( g(x) \in [-\pi, \pi] \) is:
\[
g(x) = a_0 + \sum_{n=1}^{\infty} \left[ a_n \cos(nx) + b_n \sin(nx) \right]
\]

---

\subsection*{We find the constant coefficient \( a_0 \)}

\[
a_0 = \frac{1}{2\pi} \int_{-\pi}^{\pi} g(x) \, dx = \frac{1}{2\pi} \int_{-\pi}^{\pi} \frac{x^2}{2} \, dx = \frac{1}{4\pi} \int_{-\pi}^{\pi} x^2 \, dx
\]

Note: \( x^2 \) is even, so:
\[
\int_{-\pi}^{\pi} x^2 \, dx = 2 \int_0^{\pi} x^2 \, dx
\Rightarrow a_0 = \frac{1}{4\pi} \cdot 2 \int_0^{\pi} x^2 \, dx = \frac{1}{2\pi} \int_0^{\pi} x^2 \, dx
\]

Now we integrate:
\[
\int_0^{\pi} x^2 \, dx = \left[ \frac{x^3}{3} \right]_0^{\pi} = \frac{\pi^3}{3} - 0 = \frac{\pi^3}{3}
\]

Thus:
\[
a_0 = \frac{1}{2\pi} \cdot \frac{\pi^3}{3} = \frac{\pi^2}{6}
\]

---

\subsection*{We find cosine coefficients \( a_n \)}

We start with:
\[
a_n = \frac{1}{\pi} \int_{-\pi}^{\pi} g(x) \cos(nx) \, dx = \frac{1}{\pi} \int_{-\pi}^{\pi} \frac{x^2}{2} \cos(nx) \, dx = \frac{1}{2\pi} \int_{-\pi}^{\pi} x^2 \cos(nx) \, dx
\]

Since \( x^2 \) and \( \cos(nx) \) are both even, the product \( x^2 \cos(nx) \) is even:
\[
\Rightarrow \int_{-\pi}^{\pi} x^2 \cos(nx) \, dx = 2 \int_0^{\pi} x^2 \cos(nx) \, dx
\Rightarrow a_n = \frac{1}{2\pi} \cdot 2 \int_0^{\pi} x^2 \cos(nx) \, dx = \frac{1}{\pi} \int_0^{\pi} x^2 \cos(nx) \, dx
\]

Standard integral identity:
\[
\int_0^{\pi} x^2 \cos(nx) \, dx = \frac{2\pi (-1)^n}{n^2}
\Rightarrow a_n = \frac{1}{\pi} \cdot \frac{2\pi (-1)^n}{n^2} = \frac{2(-1)^n}{n^2}
\]

---

\subsection*{We find sine coefficients \( b_n \)}

Start with:
\[
b_n = \frac{1}{\pi} \int_{-\pi}^{\pi} \frac{x^2}{2} \sin(nx) \, dx = \frac{1}{2\pi} \int_{-\pi}^{\pi} x^2 \sin(nx) \, dx
\]

Here, \( x^2 \) is even and \( \sin(nx) \) is odd \( \Rightarrow \) product is odd:
\[
\Rightarrow \int_{-\pi}^{\pi} x^2 \sin(nx) \, dx = 0 \Rightarrow b_n = 0
\]

---

\subsection*{Final Fourier Series}

We collect all terms:
\[
g(x) = a_0 + \sum_{n=1}^{\infty} a_n \cos(nx) = \frac{\pi^2}{6} + \sum_{n=1}^{\infty} \frac{2(-1)^n}{n^2} \cos(nx)
\]

We truncate to 4 cosine terms:
\[
g(x) \approx \frac{\pi^2}{6} - \frac{2}{1^2} \cos(x) + \frac{2}{2^2} \cos(2x) - \frac{2}{3^2} \cos(3x) + \frac{2}{4^2} \cos(4x)
\]
\[
= \frac{\pi^2}{6} - 2 \cos(x) + \frac{1}{2} \cos(2x) - \frac{2}{9} \cos(3x) + \frac{1}{8} \cos(4x)
\]

---

\subsection*{Bonus: Derive \( \sum_{n=1}^{\infty} \frac{1}{n^2} = \frac{\pi^2}{6} \)}
Evaluating \( g(x) \) at \( x = \pi \):
\[
g(\pi) = \frac{\pi^2}{2}
\]

From Fourier series:
\[
g(\pi) = \frac{\pi^2}{6} + \sum_{n=1}^{\infty} \frac{2(-1)^n}{n^2} \cos(n\pi)
\]

But:
\[
\cos(n\pi) = (-1)^n \Rightarrow \sum_{n=1}^{\infty} \frac{2(-1)^n}{n^2} \cdot (-1)^n = \sum_{n=1}^{\infty} \frac{2}{n^2}
\]

So:
\[
\frac{\pi^2}{2} = \frac{\pi^2}{6} + \sum_{n=1}^{\infty} \frac{2}{n^2}
\Rightarrow \frac{\pi^2}{2} - \frac{\pi^2}{6} = \sum_{n=1}^{\infty} \frac{2}{n^2}
\Rightarrow \frac{\pi^2}{3} = \sum_{n=1}^{\infty} \frac{2}{n^2}
\Rightarrow \sum_{n=1}^{\infty} \frac{1}{n^2} = \boxed{ \frac{\pi^2}{6} }
\]

\newpage
\section*{Problem 3}

Evaluate:
\[
\int_0^\infty \frac{\cos(xw) + w \sin(xw)}{1 + w^2} \, dw = g(x)
\]

---

\subsection*{We split the Integral}

We define:
\[
I_1(x) = \int_0^\infty \frac{\cos(xw)}{1 + w^2} \, dw
\quad\text{and}\quad
I_2(x) = \int_0^\infty \frac{w \sin(xw)}{1 + w^2} \, dw
\]

Then:
\[
g(x) = I_1(x) + I_2(x)
\]

---

\subsection*{We express as Fourier-Type Expansion}

Let:
\[
A(w) = \int_0^\infty g(v) \cos(vw) \, dv
\qquad
B(w) = \int_0^\infty g(v) \sin(vw) \, dv
\]

Inverse cosine-sine representation:
\[
g(x) = \int_0^\infty A(w) \cos(xw) \, dw + \int_0^\infty B(w) \sin(xw) \, dw
\]

---

\subsection*{We define \( g(x) \)}

\[
g(x) =
\begin{cases}
0, & x < 0 \\
\pi, & x = 0 \\
\pi e^{-x}, & x > 0
\end{cases}
\Rightarrow g(v) = \pi e^{-v}, \quad v > 0
\]

---

\subsection*{We find \( A(w) \)}

\[
A(w) = \int_0^\infty g(v) \cos(vw) \, dv
= \int_0^\infty \pi e^{-v} \cos(vw) \, dv
= \pi \int_0^\infty e^{-v} \cos(vw) \, dv
\]

Known result:
\[
\int_0^\infty e^{-v} \cos(vw) \, dv = \frac{1}{1 + w^2}
\]

Then:
\[
A(w) = \pi \cdot \frac{1}{1 + w^2}
= \frac{\pi}{1 + w^2}
\]

---

\subsection*{So, doing algebraic cerification of \( A(w) \)}

Given:
\[
A(w) = \frac{\pi}{1 + w^2}
\Rightarrow (1 + w^2)A(w) = \pi
\Rightarrow A(w)(1 + w^2) = \pi
\Rightarrow A(w) = \frac{\pi}{1 + w^2}
\]

Now it's verified.

---

\subsection*{We find \( B(w) \)}

\[
B(w) = \int_0^\infty g(v) \sin(vw) \, dv = \int_0^\infty \pi e^{-v} \sin(vw) \, dv = \pi \int_0^\infty e^{-v} \sin(vw) \, dv
\]

Use:
\[
\int_0^\infty e^{-v} \sin(vw) \, dv = \frac{w}{1 + w^2}
\]

Then:
\[
B(w) = \pi \cdot \frac{w}{1 + w^2}
= \frac{\pi w}{1 + w^2}
\]

---

\subsection*{Same, now algebraic verification of \( B(w) \)}

Given:
\[
B(w) = \frac{\pi w}{1 + w^2}
\Rightarrow B(w)(1 + w^2) = \pi w
\Rightarrow \frac{\pi w}{1 + w^2} \cdot (1 + w^2) = \pi w
\Rightarrow B(w) = \frac{\pi w}{1 + w^2}
\]

It's verified.

---

\subsection*{Step 8: Plug \( A(w) \), \( B(w) \) into Inverse Representation}

We start from:
\[
g(x) = \int_0^\infty A(w) \cos(xw) \, dw + \int_0^\infty B(w) \sin(xw) \, dw
\]

We substitute:
\[
= \int_0^\infty \frac{\pi}{1 + w^2} \cos(xw) \, dw + \int_0^\infty \frac{\pi w}{1 + w^2} \sin(xw) \, dw
\]

Factor \( \pi \):
\[
= \pi \left( \int_0^\infty \frac{\cos(xw)}{1 + w^2} \, dw + \int_0^\infty \frac{w \sin(xw)}{1 + w^2} \, dw \right)
\]

So:
\[
g(x) = \pi \int_0^\infty \frac{\cos(xw) + w \sin(xw)}{1 + w^2} \, dw
\]

Thus:
\[
\boxed{
\int_0^\infty \frac{\cos(xw) + w \sin(xw)}{1 + w^2} \, dw = g(x) = \pi e^{-x}, \quad x > 0
}
\]

\newpage
\section*{Problem 4}

\textbf{Goal:}
\[
\boxed{
\int_0^\infty \frac{\cos(xw)}{1 + w^2} \, dw = \frac{\pi}{2} e^{-x}, \quad x > 0
}
\]

---

\subsection*{We assume \( g(x) = e^{-x} \), for \( x > 0 \)}

Express \( g(x) \) via cosine transform:
\[
g(x) = \int_0^\infty A(w) \cos(xw) \, dw
\Rightarrow
e^{-x} = \int_0^\infty A(w) \cos(xw) \, dw
\]

---

\subsection*{Step 2: Use Inverse Cosine Transform to Find \( A(w) \)}

Inverse formula:
\[
A(w) = \frac{2}{\pi} \int_0^\infty g(u) \cos(uw) \, du
\]

Substitute \( g(u) = e^{-u} \):
\[
A(w) = \frac{2}{\pi} \int_0^\infty e^{-u} \cos(uw) \, du
\]

Define:
\[
I = \int_0^\infty e^{-u} \cos(uw) \, du
\]

---

\subsection*{We evaluate \( I = \int_0^\infty e^{-u} \cos(uw) \, du \)}

Let:
\[
f(u) = e^{-u}, \quad f'(u) = -e^{-u}
\quad \text{and} \quad
g(u) = \cos(uw), \quad g'(u) = -w \sin(uw)
\]

Integration by parts:
\[
I = \left[ \frac{e^{-u} \sin(uw)}{w} \right]_0^\infty + \frac{1}{w} \int_0^\infty e^{-u} \sin(uw) \, du
\]

But:
\[
\int_0^\infty e^{-u} \cos(uw) \, du = \Re\left( \int_0^\infty e^{-u(1 - iw)} \, du \right)
= \Re\left( \frac{1}{1 - iw} \right)
= \frac{1}{1 + w^2}
\]

Therefore:
\[
I = \int_0^\infty e^{-u} \cos(uw) \, du = \frac{1}{1 + w^2}
\]

Then:
\[
A(w) = \frac{2}{\pi} \cdot \frac{1}{1 + w^2}
\Rightarrow
A(w) = \frac{2}{\pi(1 + w^2)}
\]

---

\subsection*{We verify \( A(w) \) identity algebraically}

\[
A(w) = \frac{2}{\pi(1 + w^2)} \Rightarrow A(w)(1 + w^2) = \frac{2}{\pi}
\Rightarrow \frac{2}{\pi(1 + w^2)} = A(w)
\]

Now Verified.

---

\subsection*{We substitute \( A(w) \) into Cosine Transform of \( g(x) \)}

Recall:
\[
g(x) = \int_0^\infty A(w) \cos(xw) \, dw
\]

Then:
\[
e^{-x} = \int_0^\infty \left( \frac{2}{\pi(1 + w^2)} \right) \cos(xw) \, dw
\Rightarrow
e^{-x} = \frac{2}{\pi} \int_0^\infty \frac{\cos(xw)}{1 + w^2} \, dw
\]

---

\subsection*{We multiply Both Sides by \( \frac{\pi}{2} \)}

\[
\frac{\pi}{2} \cdot e^{-x}
= \frac{\pi}{2} \cdot \left( \frac{2}{\pi} \int_0^\infty \frac{\cos(xw)}{1 + w^2} \, dw \right)
= \int_0^\infty \frac{\cos(xw)}{1 + w^2} \, dw
\]

---

\subsection*{Final Result}

\[
\boxed{
\int_0^\infty \frac{\cos(xw)}{1 + w^2} \, dw = \frac{\pi}{2} e^{-x}, \quad x > 0
}
\]

\newpage
\section*{Problem 5}

---

\subsection*{1. Fourier Cosine Transform of \( f(x) = \frac{1}{1 + x^2} \)}

\textbf{Definition:}
\[
\mathcal{F}_c(f)(w) = \sqrt{\frac{2}{\pi}} \int_0^\infty f(x) \cos(wx) \, dx
\]

We substitute \( f(x) = \frac{1}{1 + x^2} \):
\[
\mathcal{F}_c\left( \frac{1}{1 + x^2} \right) = \sqrt{\frac{2}{\pi}} \int_0^\infty \frac{\cos(wx)}{1 + x^2} \, dx
\]

---

\textbf{Case 1: When \( w = 0 \)}

Then:
\[
\cos(0 \cdot x) = 1 \quad \Rightarrow \quad \mathcal{F}_c(w = 0) = \sqrt{\frac{2}{\pi}} \int_0^\infty \frac{1}{1 + x^2} \, dx
\]

Use:
\[
\int_0^\infty \frac{1}{1 + x^2} \, dx = \left[ \tan^{-1}(x) \right]_0^\infty = \frac{\pi}{2}
\]

Then:
\[
\mathcal{F}_c(w = 0) = \sqrt{\frac{2}{\pi}} \cdot \frac{\pi}{2}
= \frac{\sqrt{2\pi}}{2}
= \boxed{ \sqrt{ \frac{\pi}{2} } }
\]

---

\textbf{Case 2: When \( w > 0 \)}

Known identity:
\[
\int_0^\infty \frac{\cos(wx)}{1 + x^2} \, dx = \frac{\pi}{2} e^{-w}
\]

Therefore:
\[
\mathcal{F}_c\left( \frac{1}{1 + x^2} \right)
= \sqrt{\frac{2}{\pi}} \cdot \left( \frac{\pi}{2} e^{-w} \right)
= \frac{\sqrt{2} \cdot \pi \cdot e^{-w}}{2\sqrt{\pi}}
= \boxed{ \sqrt{ \frac{\pi}{2} } \cdot e^{-w} }
\]

---

\subsection*{2. Fourier Cosine Transform of \( f(x) = e^{-a x} \), where \( a > 0 \)}

\textbf{Definition:}
\[
\mathcal{F}_c(f)(w) = \sqrt{\frac{2}{\pi}} \int_0^\infty e^{-a x} \cos(wx) \, dx
\]

Instead of evaluating directly, we use differential identity:
\[
f(x) = e^{-a x}, \quad f'(x) = -a e^{-a x}, \quad f''(x) = a^2 e^{-a x}
\]

---

\textbf{Left-hand Side (LHS):}
\[
\mathcal{F}_c(f'') = \mathcal{F}_c(a^2 e^{-a x}) = a^2 \cdot \mathcal{F}_c(f)
\]

---

\textbf{Right-hand Side (RHS):}
Use identity:
\[
\mathcal{F}_c(f'') = -w^2 \cdot \mathcal{F}_c(f) - \sqrt{ \frac{2}{\pi} } \cdot f'(0)
\]

Now:
\[
f'(x) = -a e^{-a x} \Rightarrow f'(0) = -a
\Rightarrow \mathcal{F}_c(f'') = -w^2 \cdot \mathcal{F}_c(f) - \sqrt{ \frac{2}{\pi} } \cdot (-a)
= -w^2 \cdot \mathcal{F}_c(f) + \sqrt{ \frac{2}{\pi} } a
\]

---

\textbf{Equating LHS and RHS:}
\[
a^2 \cdot \mathcal{F}_c(f) = -w^2 \cdot \mathcal{F}_c(f) + \sqrt{ \frac{2}{\pi} } a
\]

Group terms:
\[
a^2 \cdot \mathcal{F}_c(f) + w^2 \cdot \mathcal{F}_c(f) = \sqrt{ \frac{2}{\pi} } a
\Rightarrow (a^2 + w^2) \cdot \mathcal{F}_c(f) = \sqrt{ \frac{2}{\pi} } a
\]

So, we solve for \( \mathcal{F}_c(f) \):
\[
\mathcal{F}_c(f) = \frac{ \sqrt{ \frac{2}{\pi} } a }{ a^2 + w^2 }
= \boxed{ \frac{ a \sqrt{ \frac{2}{\pi} } }{ a^2 + w^2 } }
\]

---

\subsection*{3. Alternative Equivalent Form – Multiply by \( w/w \)}

\[
\mathcal{F}_c(f) = \frac{ a \sqrt{2/\pi} }{ a^2 + w^2 }
\]

We multiply numerator and denominator by \( w \):
\[
= \frac{ w \cdot a \cdot \sqrt{2/\pi} }{ w(a^2 + w^2) }
= \frac{w}{ \frac{w \cdot \sqrt{2/\pi} }{ a^2 + w^2 } }
\]

\textbf{Equivalent Form:}
\[
\boxed{
\mathcal{F}_c(e^{-a x}) = \frac{w}{ \left( \frac{ w \sqrt{2/\pi} }{ a^2 + w^2 } \right) }
}
\]

\end{document}
