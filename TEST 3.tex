\documentclass{article}
\usepackage{amsmath,amsfonts}
\usepackage{tcolorbox}
\usepackage{listings}
\usepackage[margin=2.5cm]{geometry}
\usepackage{amsmath, amssymb}
\usepackage{graphicx}
\usepackage{titlesec}
\usepackage{hyperref}
\usepackage{xcolor}

\begin{document}

\date{}

\begin{titlepage}
    \centering
    \includegraphics[width=0.5\textwidth]{uwm-logo.png} \\[1cm]
    
    {\Large \textbf{Mechanical Engineering Department}}\\[0.5em]
    {\large ME604/704 - Advanced Engineering Mathematics II}\\[0.5em]
    {\large Instructor: Dr. Istvan Lauko}\\[0.5em]
    {\large Spring 2025}\\[2cm]
    
    {\large \textbf{Class:} MATH 604}\\[0.5em]
    {\large \textbf{} Test III}\\[0.5em]
    {\large \textbf{Name:} Abdallah Benelmadjat}\\[0.5em]
    {\large Find the \LaTeX\ GitHub source code by clicking \href{https://github.com/abdallah-benelmadjat/MATH-602/blob/main/HW4.tex}{\textcolor{blue}{here}}}\\
\end{titlepage}

\newpage

%%%%%%%%%%%%%%%%%%%%%%%%%%%%%%%%%%%%%%%%%%%%%%%%%%%%%%%%%%%%%%%%%%%%%%%%%%%%%%%%
% 1.  Trial space and approximate solution
%%%%%%%%%%%%%%%%%%%%%%%%%%%%%%%%%%%%%%%%%%%%%%%%%%%%%%%%%%%%%%%%%%%%%%%%%%%%%%%%
\section*{Problem 1}

\vspace{0.5em}
\textbf{Continuous problem}
\[
\begin{cases}
x^{2}y''+4xy'+2y=-2\ln x, & 1\le x\le 2,\\
y(1)=-0.5,\\
y(2)=\ln 2.
\end{cases}
\]

\textbf{Grid}
\[
h=\dfrac{1}{N}, 
\qquad
x_i=1+(i-1)h,\; i=1,\dots,N+1.
\]

\textbf{Central differences}
\[
y'_i=\dfrac{y_{i+1}-y_{i-1}}{2h},
\qquad
y''_i=\dfrac{y_{i+1}-2y_i+y_{i-1}}{h^{2}}.
\]

\textbf{Discrete equation at node \(i\) \((i=2,\dots,N)\)}
\[
\left(\frac{x_i^{2}}{h^{2}}+\frac{2x_i}{h}\right)y_{i+1}
+\left(-2\frac{x_i^{2}}{h^{2}}+2\right)y_{i}
+\left(\frac{x_i^{2}}{h^{2}}-\frac{2x_i}{h}\right)y_{i-1}
=-2\ln x_i.
\]

\textbf{Coefficient notation}
\[
a_i=\frac{x_i^{2}}{h^{2}}-\frac{2x_i}{h},\quad
b_i=-2\frac{x_i^{2}}{h^{2}}+2,\quad
c_i=\frac{x_i^{2}}{h^{2}}+\frac{2x_i}{h},\quad
d_i=-2\ln x_i.
\]

\textbf{Interior stencil}
\[
a_i\,y_{i-1}+b_i\,y_i+c_i\,y_{i+1}=d_i,\quad i=2,\dots,N.
\]

\textbf{Boundary values}
\[
y_1=-0.5,\qquad y_{N+1}=\ln 2.
\]

\textbf{Linear system}
\[
\underbrace{
\begin{bmatrix}
b_{2} & c_{2} & 0 & \cdots & 0 \\
a_{3} & b_{3} & c_{3} & \ddots & \vdots \\
0 & \ddots & \ddots & \ddots & 0 \\
\vdots & \ddots & a_{N-1} & b_{N-1} & c_{N-1}\\
0 & \cdots & 0 & a_{N} & b_{N}
\end{bmatrix}}_{\displaystyle A\in\mathbb{R}^{N-1\times N-1}}
%
\underbrace{
\begin{bmatrix}
y_{2}\\ y_{3}\\ \vdots\\ y_{N-1}\\ y_{N}
\end{bmatrix}}_{\displaystyle \mathbf y}
=
\underbrace{
\begin{bmatrix}
d_{2}-a_{2}y_{1}\\
d_{3}\\
\vdots\\
d_{N-1}\\
d_{N}-c_{N}y_{N+1}
\end{bmatrix}}_{\displaystyle \mathbf b}.
\]

\section*{MATLAB Code}

\lstset{
    language=Matlab,
    basicstyle=\ttfamily\footnotesize,
    keywordstyle=\color{blue},
    commentstyle=\color{gray},
    stringstyle=\color{red},
    numbers=left,
    numberstyle=\tiny\color{gray},
    stepnumber=1,
    numbersep=5pt,
    frame=single,
    breaklines=true,
    showstringspaces=false
}

\begin{tcolorbox}[title=MATLAB code to solve using Jordan method:]
\begin{lstlisting}
clc
clear
h = 0.1;
x = 1:h:2;
N = length(x);
y = zeros(N, 1);
y(1) = -0.5;
y(N) = log(2);
for i = 2:N-1
    p = x(i)/h;
    y(i) = (((-2*p) - (p^2)) * y(i+1) + ((2*p) - (p^2)) * y(i-1) - 2*log(x(i))) / (2 - (2*(x(i)^2))/(h^2));
end
\end{lstlisting}
\end{tcolorbox}

\newpage


\section*{Problem 2}

\paragraph{Continuous model}
\[
T_t=c^{2}T_{xx}, \qquad c^{2}=3,\; 0\le x\le L=2,\; t\ge0.
\]

\[
T(0,t)=\sin(\pi t), \qquad
T(L,t)=\sin(2\pi t), \qquad
T(x,0)=0.
\]

\paragraph{Grid}
\[
\Delta x=0.2,\quad x_i=i\Delta x,\; i=0,\dots,n+1,\; n=\tfrac{L}{\Delta x}-1=9.
\]
\[
\Delta t=0.25,\quad t_j=j\Delta t,\; j=0,\dots,m,\; m=\tfrac{10}{\Delta t}=40.
\]
\[
r=\frac{c^{2}\Delta t}{\Delta x^{2}}
     =\frac{3\times0.25}{0.2^{2}}
     =18.75.
\]
\emph{Remark:}\; CN is unconditionally stable, so the large value of \(r\) is allowed, but accuracy improves if \(\Delta x\) is reduced.

\paragraph{Interior stencil (\(1\le i\le n,\,0\le j\le m-1\))}
\[
(1+r)T_i^{j+1}-\frac{r}{2}T_{i+1}^{j+1}-\frac{r}{2}T_{i-1}^{j+1}
=(1-r)T_i^{j}+\frac{r}{2}T_{i+1}^{j}+\frac{r}{2}T_{i-1}^{j}.
\]

\paragraph{Coefficients}
\[
a=-\frac{r}{2},\qquad b=1+r,\qquad c=-\frac{r}{2}.
\]

\paragraph{Linear system at each step}
\[
A\mathbf T^{\,j+1}=\mathbf B^{j},
\]
\[
A=
\begin{bmatrix}
b & c & 0 & \cdots & 0\\
a & b & c & \ddots & \vdots\\
0 & \ddots & \ddots & \ddots & 0\\
\vdots & \ddots & a & b & c\\
0 & \cdots & 0 & a & b
\end{bmatrix}_{n\times n}.
\]

\[
\mathbf B^{j}=a\mathbf T^{j-1}+ (b-2r)\mathbf T^{j}+c\mathbf T^{j+1}
 +\begin{bmatrix}
 a\bigl[T_0^{j}+T_0^{\,j+1}\bigr]\\[2pt]
 0\\ \vdots \\ 0\\[2pt]
 c\bigl[T_{n+1}^{j}+T_{n+1}^{\,j+1}\bigr]
 \end{bmatrix}.
\]

\paragraph{Boundary values inserted every step}
\[
T_0^{\,j}=\sin(\pi t_j),\qquad
T_{n+1}^{\,j}=\sin(2\pi t_j).
\]


\section*{MATLAB Code}

\lstset{
    language=Matlab,
    basicstyle=\ttfamily\footnotesize,
    keywordstyle=\color{blue},
    commentstyle=\color{gray},
    stringstyle=\color{red},
    numbers=left,
    numberstyle=\tiny\color{gray},
    stepnumber=1,
    numbersep=5pt,
    frame=single,
    breaklines=true,
    showstringspaces=false
}

\begin{tcolorbox}[title=Crank-Nicolson Method MATLAB Code for Problem 2]
\begin{lstlisting}
% code to simulate problem 2
clc, clear;
a = 3;
L = 2;
tf = 10;
del_t = 0.25;
del_x = 0.2;
d = a * del_t / (del_x^2);
m = tf / del_t;
n = L / del_x - 1;
x = (del_x:del_x:L-del_x);
t = (del_t:del_t:tf);
boun_left_temp = zeros(1, m + 1);
boun_right_temp = zeros(1, m + 1);
boun_left_temp(1) = 0;
boun_right_temp(1) = 0;
for i = 2:1:m + 1
    boun_left_temp(i) = sin(pi * t(i - 1));
    boun_right_temp(i) = sin(2 * pi * t(i - 1));
end
ini_temp = sin(pi * x / L);
m_size = m * n;
A = eye(m_size);
A = A * 2 * (1 + d);
for i = 2:m_size
    A(i, i - 1) = -d;
end
for i = n + 1:n:m_size - (n - 1)
    A(i, i - 1) = 0;
end
for i = 1:m_size - 1
    A(i, i + 1) = -d;
end
for i = n:n:m_size - n
    A(i, i + 1) = 0;
end
for i = n + 1:m_size
    A(i, i - 1) = -2 * (1 - d);
end
for i = n + 2:m_size
    A(i, i - n - 1) = -d;
end
for i = 2 * n + 1:n:m_size - n + 1
    A(i, i - (n + 1)) = 0;
end
for i = n + 1:m_size
    A(i, i - (n - 1)) = -d;
end
for i = 2 * n:n:m_size
    A(i, i - (n - 1)) = 0;
end
B = zeros(m_size, 1);
B(1) = (2 * (1 - d) * ini_temp(1)) + (d * boun_left_temp(1)) + (d * boun_left_temp(2)) + (d * ini_temp(2));
B(n) = (d * boun_right_temp(1)) + (d * boun_right_temp(2)) + (d * ini_temp(n - 1)) + (2 * (1 - d) * ini_temp(n));
for i = 2:n-1
    B(i) = (d * ini_temp(i - 1)) + (2 * (1 - d) * ini_temp(i)) + (d * ini_temp(i + 1));
end
j = 2;
for i = n + 1:n:m_size - (n - 1)
    B(i) = (d * boun_left_temp(j)) + (d * boun_left_temp(j - 1));
    j = j + 1;
end
j = 2;
for i = 2 * n:n:m_size
    B(i) = (d * boun_right_temp(j)) + (d * boun_right_temp(j - 1));
    j = j + 1;
end
Temp_dist = inv(A) * B;
l = 1;
for i = 1:m
    for j = 1:n
        T(i, j) = Temp_dist(l);
        l = l + 1;
    end
end
figure(1)
[C, h] = contour(x, t, T, 'linewidth', 1.1);
set(h, 'ShowText', 'on', 'TextStep', get(h, 'LevelStep') * 2)
colormap jet
title('Temp. contour as a function of space and time')
xlabel('Space x (cm)')
ylabel('Time [s]')
H = legend('Temp. Contours');
set(H, 'Interpreter', 'Tex')
grid on
figure(2)
surf(x, t, T)
xlabel('Space x (cm)')
ylabel('Time t (s)')
zlabel('Temperature T (C)')
\end{lstlisting}
\end{tcolorbox}


\newpage

\section*{Problem 3}

\textbf{Basis choice}  
\[
\phi_j(x)=x^{\,j}-x^{\,j+1},\qquad j=1,\dots ,N .
\]
Each $\phi_j$ satisfies $\phi_j(0)=\phi_j(1)=0$.

\textbf{Expansion}  
\[
\tilde u(x)=\sum_{j=1}^{N}c_j\phi_j(x).
\]

%%%%%%%%%%%%%%%%%%%%%%%%%%%%%%%%%%%%%%%%%%%%%%%%%%%%%%%%%%%%%%%%%%%%%%%%%%%%%%%%
% 2. Differential operator and residual
%%%%%%%%%%%%%%%%%%%%%%%%%%%%%%%%%%%%%%%%%%%%%%%%%%%%%%%%%%%%%%%%%%%%%%%%%%%%%%%%

\textbf{Operator}  
\[
L=\dfrac{d^{\,2}}{dx^{2}}.
\]

\textbf{Residual}  
\[
R(x)=L\tilde u(x)-\lambda \tilde u(x)=\sum_{j=1}^{N}c_j\phi_j''(x)-\lambda\sum_{j=1}^{N}c_j\phi_j(x).
\]

%%%%%%%%%%%%%%%%%%%%%%%%%%%%%%%%%%%%%%%%%%%%%%%%%%%%%%%%%%%%%%%%%%%%%%%%%%%%%%%%
% 3. Galerkin condition
%%%%%%%%%%%%%%%%%%%%%%%%%%%%%%%%%%%%%%%%%%%%%%%%%%%%%%%%%%%%%%%%%%%%%%%%%%%%%%%%

\textbf{Inner product}  
\[
\langle f,g\rangle=\int_{0}^{1}f(x)\,g(x)\,dx .
\]

\textbf{Orthogonality requirement}  
\[
\langle\phi_i,R\rangle=0,\qquad i=1,\dots ,N .
\]

Insert $R$:

\[
\sum_{j=1}^{N}c_j\,\langle\phi_i,\phi_j''\rangle
-
\lambda
\sum_{j=1}^{N}c_j\,\langle\phi_i,\phi_j\rangle=0.
\]

%%%%%%%%%%%%%%%%%%%%%%%%%%%%%%%%%%%%%%%%%%%%%%%%%%%%%%%%%%%%%%%%%%%%%%%%%%%%%%%%
% 4. Mass and stiffness matrices
%%%%%%%%%%%%%%%%%%%%%%%%%%%%%%%%%%%%%%%%%%%%%%%%%%%%%%%%%%%%%%%%%%%%%%%%%%%%%%%%

Define
\[
G_{ij}=\langle\phi_i,\phi_j\rangle,
\qquad
S_{ij}=\langle\phi_i',\phi_j'\rangle
\;(\;= -\langle\phi_i,\phi_j''\rangle\;\text{by parts}\;).
\]

Galerkin system  
\[
S\mathbf c=\lambda G\mathbf c,\qquad
\mathbf c=(c_1,\dots ,c_N)^{\mathrm T}.
\]

%%%%%%%%%%%%%%%%%%%%%%%%%%%%%%%%%%%%%%%%%%%%%%%%%%%%%%%%%%%%%%%%%%%%%%%%%%%%%%%%
% 5. Exact algebra for G_{ij}
%%%%%%%%%%%%%%%%%%%%%%%%%%%%%%%%%%%%%%%%%%%%%%%%%%%%%%%%%%%%%%%%%%%%%%%%%%%%%%%%

\subsection*{Matrix $G$ (``mass'')}

\begin{align*}
G_{ij}
&=\int_{0}^{1}(x^{\,i}-x^{\,i+1})(x^{\,j}-x^{\,j+1})\,dx\\[4pt]
&=\int_{0}^{1}x^{\,i+j}\,dx
-2\int_{0}^{1}x^{\,i+j+1}\,dx
+\int_{0}^{1}x^{\,i+j+2}\,dx.
\end{align*}

Using $\displaystyle\int_{0}^{1}x^{k}\,dx=\frac{1}{k+1}$:

\[
\boxed{\,G_{ij}
=\frac{1}{i+j+1}-\frac{2}{i+j+2}+\frac{1}{i+j+3}\,}.
\]

%%%%%%%%%%%%%%%%%%%%%%%%%%%%%%%%%%%%%%%%%%%%%%%%%%%%%%%%%%%%%%%%%%%%%%%%%%%%%%%%
% 6. Exact algebra for S_{ij}
%%%%%%%%%%%%%%%%%%%%%%%%%%%%%%%%%%%%%%%%%%%%%%%%%%%%%%%%%%%%%%%%%%%%%%%%%%%%%%%%

\subsection*{Matrix $S$ (``stiffness'')}

First derivative  
\[
\phi_j'(x)=j\,x^{\,j-1}-(j+1)\,x^{\,j}.
\]

Product in the integrand  
\[
\phi_i'(x)\,\phi_j'(x)
=i\,j\,x^{\,i+j-2}
-(i\,j+i(j+1))\,x^{\,i+j-1}
+(i+1)(j+1)\,x^{\,i+j}.
\]

Integrate term–by–term:

\[
\int_{0}^{1}x^{\,i+j-2}\,dx=\frac{1}{i+j-1},
\quad
\int_{0}^{1}x^{\,i+j-1}\,dx=\frac{1}{i+j},
\quad
\int_{0}^{1}x^{\,i+j}\,dx=\frac{1}{i+j+1}.
\]

Collect:

\[
\boxed{\,S_{ij}=
\frac{i\,j}{i+j-1}
-\frac{2\,i\,j+i+j}{i+j}
+\frac{(i+1)(j+1)}{i+j+1}\,}.
\]

%%%%%%%%%%%%%%%%%%%%%%%%%%%%%%%%%%%%%%%%%%%%%%%%%%%%%%%%%%%%%%%%%%%%%%%%%%%%%%%%
% 7. Explicit $4\times4$ matrices (example $N=4$)
%%%%%%%%%%%%%%%%%%%%%%%%%%%%%%%%%%%%%%%%%%%%%%%%%%%%%%%%%%%%%%%%%%%%%%%%%%%%%%%%

For $N=4$ evaluate $G_{ij},\,S_{ij}$ using the formulas.  
\vspace{4pt}

$G=$
\[
\begin{bmatrix}
\frac{1}{3}-\frac{2}{4}+\frac{1}{5} & \frac{1}{4}-\frac{2}{5}+\frac{1}{6} & \frac{1}{5}-\frac{2}{6}+\frac{1}{7} & \frac{1}{6}-\frac{2}{7}+\frac{1}{8}\\
\frac{1}{4}-\frac{2}{5}+\frac{1}{6} & \frac{1}{5}-\frac{2}{6}+\frac{1}{7} & \frac{1}{6}-\frac{2}{7}+\frac{1}{8} & \frac{1}{7}-\frac{2}{8}+\frac{1}{9}\\
\frac{1}{5}-\frac{2}{6}+\frac{1}{7} & \frac{1}{6}-\frac{2}{7}+\frac{1}{8} & \frac{1}{7}-\frac{2}{8}+\frac{1}{9} & \frac{1}{8}-\frac{2}{9}+\frac{1}{10}\\
\frac{1}{6}-\frac{2}{7}+\frac{1}{8} & \frac{1}{7}-\frac{2}{8}+\frac{1}{9} & \frac{1}{8}-\frac{2}{9}+\frac{1}{10}& \frac{1}{9}-\frac{2}{10}+\frac{1}{11}
\end{bmatrix}.
\]

$S=$
\[
\begin{bmatrix}
\frac{1}{1}-\frac{4}{2}+\frac{4}{3} & 
\frac{2}{2}-\frac{7}{3}+\frac{6}{4} & 
\frac{3}{3}-\frac{10}{4}+\frac{8}{5} & 
\frac{4}{4}-\frac{13}{5}+\frac{10}{6}\\
\frac{2}{2}-\frac{7}{3}+\frac{6}{4} &
\frac{4}{3}-\frac{12}{4}+\frac{9}{5} &
\frac{6}{4}-\frac{17}{5}+\frac{12}{6}&
\frac{8}{5}-\frac{22}{6}+\frac{15}{7}\\
\frac{3}{3}-\frac{10}{4}+\frac{8}{5}&
\frac{6}{4}-\frac{17}{5}+\frac{12}{6}&
\frac{9}{5}-\frac{22}{6}+\frac{16}{7}&
\frac{12}{6}-\frac{27}{7}+\frac{20}{8}\\
\frac{4}{4}-\frac{13}{5}+\frac{10}{6}&
\frac{8}{5}-\frac{22}{6}+\frac{15}{7}&
\frac{12}{6}-\frac{27}{7}+\frac{20}{8}&
\frac{16}{7}-\frac{32}{8}+\frac{25}{9}
\end{bmatrix}.
\]

(Values left unevaluated for transparency; insert decimals in software.)

%%%%%%%%%%%%%%%%%%%%%%%%%%%%%%%%%%%%%%%%%%%%%%%%%%%%%%%%%%%%%%%%%%%%%%%%%%%%%%%%
% 8. Generalized eigenproblem
%%%%%%%%%%%%%%%%%%%%%%%%%%%%%%%%%%%%%%%%%%%%%%%%%%%%%%%%%%%%%%%%%%%%%%%%%%%%%%%%

Solve 
\[
S\mathbf c^{(k)}=\lambda_k\,G\mathbf c^{(k)},\qquad k=1,\dots ,N,
\]
e.g.\ \verb|[V,D]=eig(S,G)| in MATLAB / Octave.  

\textbf{Approximate eigenfunction}  
\[
u_k(x)=\sum_{j=1}^{N}c^{(k)}_j\phi_j(x).
\]

%%%%%%%%%%%%%%%%%%%%%%%%%%%%%%%%%%%%%%%%%%%%%%%%%%%%%%%%%%%%%%%%%%%%%%%%%%%%%%%%
% 9. Brief comments
%%%%%%%%%%%%%%%%%%%%%%%%%%%%%%%%%%%%%%%%%%%%%%%%%%%%%%%%%%%%%%%%%%%%%%%%%%%%%%%%

\section*{MATLAB Code}

\lstset{
    language=Matlab,
    basicstyle=\ttfamily\footnotesize,
    keywordstyle=\color{blue},
    commentstyle=\color{gray},
    stringstyle=\color{red},
    numbers=left,
    numberstyle=\tiny\color{gray},
    stepnumber=1,
    numbersep=5pt,
    frame=single,
    breaklines=true,
    showstringspaces=false
}

\begin{tcolorbox}[title=Galerkin Method MATLAB Code]
\begin{lstlisting}
clc;
clear;
N = 4;
G = zeros(N, N);
S = zeros(N, N);
for i = 1:N
    for j = 1:N
        G(i, j) = -1 / (i + j + 1) - 2 / (i + j + 2) + 1 / (i + j + 3);
        S(i, j) = -i * j / (i + j - 1) ...
                  - (2 * i * j + i + j) / (i + j) ...
                  + (i + 1) * (j + 1) / (i + j + 1);
    end
end
disp('Gram Matrix G:');
disp(G);
disp('Stiffness Matrix S:');
disp(S);
Z = inv(G) * S;
[C, Lambda] = eig(Z);
disp('Eigenvalues (Lambda):');
disp(diag(Lambda));
disp('Eigenvectors (C):');
disp(C);
\end{lstlisting}
\end{tcolorbox}

\section*{Results}

\begin{tcolorbox}[title=Gram Matrix $G$]
\[
G =
\begin{bmatrix}
-0.6333 & -0.4833 & -0.3905 & -0.3274 \\
-0.4833 & -0.3905 & -0.3274 & -0.2817 \\
-0.3905 & -0.3274 & -0.2817 & -0.2472 \\
-0.3274 & -0.2817 & -0.2472 & -0.2202
\end{bmatrix}
\]
\end{tcolorbox}

\vspace{0.5cm}

\begin{tcolorbox}[title=Stiffness Matrix $S$]
\[
S =
\begin{bmatrix}
-1.6667 & -1.8333 & -1.9000 & -1.9333 \\
-1.8333 & -2.5333 & -2.9000 & -3.1238 \\
-1.9000 & -2.9000 & -3.5143 & -3.9286 \\
-1.9333 & -3.1238 & -3.9286 & -4.5079
\end{bmatrix}
\]
\end{tcolorbox}

\vspace{0.5cm}

\begin{tcolorbox}[title=Eigenvalues $\Lambda$]
\[
\Lambda =
\begin{bmatrix}
277.4339 \\
65.0520 \\
1.8792 \\
21.5381
\end{bmatrix}
\]
\end{tcolorbox}

\vspace{0.5cm}

\begin{tcolorbox}[title=Eigenvectors $C$]
\[
C =
\begin{bmatrix}
0.0668  & -0.1142 & 0.7502  & 0.4688 \\
-0.4258 & 0.5433  & -0.5924 & -0.7631 \\
0.7876  & -0.7637 & 0.2764  & -0.1776 \\
-0.4402 & 0.3294  & -0.1000 & 0.4080
\end{bmatrix}
\]
\end{tcolorbox}

\newpage

\section*{Problem 4}

\subsection*{Uniform mesh}

\[
x_i=i\Delta x,\qquad \Delta x=\dfrac1N,\qquad i=0,1,\dots ,N
\]

\subsection*{Piecewise linear basis}

\[
\phi_j(x)=
\begin{cases}
\dfrac{x-x_{j-1}}{\Delta x}, & x_{j-1}<x<x_j,\\[4pt]
\dfrac{x_{j+1}-x}{\Delta x}, & x_j<x<x_{j+1},\\[4pt]
0, & \text{otherwise},
\end{cases}
\qquad j=1,\dots ,N-1
\]

\[
\phi_j'(x)=
\begin{cases}
\dfrac1{\Delta x}, & x_{j-1}<x<x_j,\\[6pt]
-\dfrac1{\Delta x}, & x_j<x<x_{j+1},\\[6pt]
0, & \text{otherwise}.
\end{cases}
\]

\subsection*{Approximation}

\[
\tilde u(x)=\sum_{j=1}^{N-1}c_j\phi_j(x).
\]

\bigskip
%%%%%%%%%%%%%%%%%%%%%%%%%%%%%%%%%%%%%%%%%%%%%%%%%%%%%%%%%%%%%%%%%%%%%%%%%%%%%%%%
\section*{Part (a)}

\[
u''+u'=5,\qquad u(0)=u(1)=0.
\]

\subsection*{Inner products}

\vspace{-2ex}
\[
\int_0^1\phi_i'(x)\phi_j'(x)\,dx=
\begin{cases}
\dfrac{2}{\Delta x}, & i=j,\\[6pt]
-\dfrac{1}{\Delta x}, & |i-j|=1,\\[6pt]
0, & \text{otherwise};
\end{cases}
\]

\[
\int_0^1\phi_i(x)\phi_j'(x)\,dx=
\begin{cases}
1, & i=j,\\[6pt]
-\dfrac12, & |i-j|=1,\\[6pt]
0, & \text{otherwise};
\end{cases}
\]

\[
\int_0^1\phi_i(x)\,5\,dx=5\Delta x,\qquad i=1,\dots ,N-1.
\]

\subsection*{Matrices and vector}

\[
K_{ij}=\int_0^1\phi_i'(x)\phi_j'(x)\,dx,\qquad
M_{ij}=\int_0^1\phi_i(x)\phi_j'(x)\,dx,\qquad
F_i=\int_0^1\phi_i(x)\,5\,dx.
\]

\[
K=\frac1{\Delta x}
\begin{bmatrix}
 2&-1& 0&\dots &0\\
-1& 2&-1&\dots &0\\
 0&-1& 2&\dots &0\\
\vdots&\vdots&\vdots&\ddots&-1\\
 0& 0& 0&-1& 2
\end{bmatrix}_{(N-1)\times(N-1)},
\quad
M=
\begin{bmatrix}
 1&-\tfrac12&0&\dots &0\\
-\tfrac12&1&-\tfrac12&\dots &0\\
0&-\tfrac12&1&\dots &0\\
\vdots&\vdots&\vdots&\ddots&-\tfrac12\\
0&0&0&-\tfrac12&1
\end{bmatrix},
\quad
F=5\Delta x\,\mathbf 1.
\]

\subsection*{Linear system}

\[
(-K+M)c=F,\qquad c\in\mathbb R^{N-1}.
\]

\subsection*{Exact solution}

\[
u_{\mathrm{exact}}(x)=\frac{5}{1-e^{-1}}\bigl(-1+e^{-x}\bigr)+5x.
\]

%%%%%%%%%%%%%%%%%%%%%%%%%%%%%%%%%%%%%%%%%%%%%%%%%%%%%%%%%%%%%%%%%%%%%%%%%%%%%%%%
\section*{Part (b)}

\[
u''+u'=5,\qquad u'(0)=0,\;u(1)=0.
\]

\subsection*{Modified inner products}

First basis function has support on $(0,x_1)$ only on the left side in the weak form with Neumann condition.

\[
\int_0^1\phi_i'(x)\phi_j'(x)\,dx=
\begin{cases}
\dfrac{1}{\Delta x}, & i=j=1,\\[6pt]
\dfrac{2}{\Delta x}, & i=j=2,\dots ,N-1,\\[6pt]
-\dfrac{1}{\Delta x}, & |i-j|=1,\\[6pt]
0, & \text{otherwise};
\end{cases}
\]

\[
\int_0^1\phi_i(x)\phi_j'(x)\,dx=
\begin{cases}
\dfrac12, & i=j=1,\\[6pt]
1, & i=j=2,\dots ,N-1,\\[6pt]
-\dfrac12, & |i-j|=1,\\[6pt]
0, & \text{otherwise};
\end{cases}
\]

\[
\int_0^1\phi_1(x)\,5\,dx=\tfrac32\cdot5\Delta x,\qquad
\int_0^1\phi_i(x)\,5\,dx=5\Delta x,\;i=2,\dots ,N-1.
\]

\subsection*{Matrices and vector}

\[
\widehat K_{11}=\frac1{\Delta x},\quad
\widehat M_{11}=\frac12,\quad
\widehat F_1=\frac32\cdot5\Delta x;
\]
remaining rows/columns identical to those of $K$, $M$, $F$ in part (a).

\subsection*{Linear system}

\[
(-\widehat K+\widehat M)c=\widehat F.
\]

\subsection*{Exact solution}

\[
u_{\mathrm{exact}}(x)= -5e^{-1}-5+5e^{-x}+5x.
\]

\newpage

\section*{MATLAB Code}

\lstset{
    language=Matlab,
    basicstyle=\ttfamily\footnotesize,
    keywordstyle=\color{blue},
    commentstyle=\color{gray},
    stringstyle=\color{red},
    numbers=left,
    numberstyle=\tiny\color{gray},
    stepnumber=1,
    numbersep=5pt,
    frame=single,
    breaklines=true,
    showstringspaces=false
}

\begin{tcolorbox}[title=Piecewise‐linear FEM MATLAB Code]
\begin{lstlisting}
clc; clear;
nElem = 199;                 
fVal  = 5;                   
x  = linspace(0,1,nElem+1).';
dx = x(2)-x(1);
N  = numel(x);

e  = ones(nElem-1,1);
K  = spdiags([-e  2*e  -e],[-1 0 1],nElem-1,nElem-1)/dx;
M  = spdiags([-0.5*e  e  -0.5*e],[-1 0 1],nElem-1,nElem-1);

A  = -K + M;                       
F  = fVal*dx*ones(nElem-1,1);      

c  = A\F;                          
u          = zeros(N,1);           
u(2:end-1) = c;

uExact = 5*x + (5/(1-exp(-1)))*(exp(-x)-1);

errInf = max(abs(u-uExact));
errL2  = sqrt( trapz(x,(u-uExact).^2) );
fprintf('max-norm error = %.4e\n',errInf);
fprintf('L2-norm  error = %.4e\n',errL2);

figure
plot(x,u,x,uExact,'r','LineWidth',1.8)
legend('FEM','Exact','Location','best')
xlabel('x'); ylabel('u(x)')
title(sprintf('Piecewise-linear FEM   (N = %d elements)',nElem))
grid on
\end{lstlisting}
\end{tcolorbox}

\begin{figure}[h!]
    \centering
    \includegraphics[width=0.8\textwidth]{image1.jpg}
    \caption{FEM Approximation vs. Exact Solution for $u'' + u' = 5$}
    \label{fig:fem_solution}
\end{figure}

\newpage

\section*{MATLAB Code}

\lstset{
    language=Matlab,
    basicstyle=\ttfamily\footnotesize,
    keywordstyle=\color{blue},
    commentstyle=\color{gray},
    stringstyle=\color{red},
    numbers=left,
    numberstyle=\tiny\color{gray},
    stepnumber=1,
    numbersep=5pt,
    frame=single,
    breaklines=true,
    showstringspaces=false
}

\begin{tcolorbox}[title=Finite Element Method (Piecewise Linear Basis)]
\begin{lstlisting}
clc
clear
u0=0; 
u1=0; 
f=5; 
a=0; 
b=1; 
x=linspace(a,b,200); 
N=length(x); 
dx=(b-a)/(N-1);
u=zeros(1,N);
u(1)=u0;
u(N)=u1;

G=zeros(N-2,N-2); 
for i=1:N-2
    if i==1
        G(i,i)=-1/dx;
    else
        G(i,i)=-2/dx;
    end
    if i<N-2
        G(i,i+1)=1/dx;
    end
    if i>1
        G(i,i-1)=1/dx;
    end
end

S=zeros(N-2,N-2); 
for i=1:N-2
    if i==1
        S(i,i)=0.5;
    else
        S(i,i)=1;
    end
    if i<N-2
        S(i,i+1)=-0.5;
    end
    if i>1
        S(i,i-1)=-0.5;
    end
end

F=zeros(N-2,1);
F=dx*f+F;
F(1)=0.5*dx*f+F(1);
A=G+S;
C=A\F;

for i=2:N-1
    u(i)=C(i-1);
end

u(1)=u(2);

for i=1:N
    y(i)=5*(x(i)+exp(-x(i))-exp(-1)-1);
end

figure(1) 
plot(x,u,x,y,'red','linewidth',2)
title('BVP Exact and Finit Elemnt Solution', 'FontSize', 16);
xlabel('x', 'FontSize', 14);
ylabel('u', 'FontSize', 14);
\end{lstlisting}
\end{tcolorbox}

\begin{figure}[h!]
    \centering
    \includegraphics[width=0.8\textwidth]{image2.jpg}
    \caption{BVP Exact and Finit Elemnt Solution for $u'' + u' = 5$}
    \label{fig:fem_solution}
\end{figure}

\newpage

\section*{Problem 5}

Given boundary value problem:
\[
x^2 y'' - x y' + (\pi^2 + 1)y = 0, \quad 1 \leq x \leq 2,
\]
\[
y(1) = 0, \quad y(2) = 2 \sin(\pi \ln 2), \quad y(2) = 1.643.
\]

\subsection*{Solution}

Domain subdivision:
\[
x_i = a + \frac{i}{N} (b - a), \quad i = 0, 1, \dots, N,
\]
where $a = 1$, $b = 2$, so
\[
x_i = 1 + \frac{i}{N}.
\]

\subsection*{Basis Functions}

Define hat functions:
\[
\varphi_j (x) =
\begin{cases}
N \left( x - \frac{j-1}{N} \right), & \frac{j-1}{N} < x < \frac{j}{N}, \\
N \left( \frac{j+1}{N} - x \right), & \frac{j}{N} < x < \frac{j+1}{N}, \\
0, & \text{otherwise}.
\end{cases}
\]
for $j = 1, 2, \dots, N-1$.

\subsection*{Piecewise Linear Approximate Solution}

\[
\tilde{u}(x) = \sum_{j=1}^{N-1} c_j \varphi_j(x).
\]

\subsection*{Weak Formulation}

Given equation:
\[
x^2 y'' - x y' + (\pi^2 + 1)y = 0,
\]
Multiply by test function $w$:
\[
\int_1^2 x^2 w y'' \, dx - \int_1^2 x w y' \, dx + \int_1^2 (\pi^2 + 1) w y \, dx = 0.
\]
Integrate by parts:
\[
\int_1^2 -x^2 w' y' \, dx + \left[ x^2 w y' \right]_1^2 - \int_1^2 x w y' \, dx + \int_1^2 (\pi^2 + 1) w y \, dx = 0.
\]
Boundary term:
\[
\left[ x^2 w y' \right]_1^2 = x^2 w y' \bigg|_1^2.
\]
Global system:
\[
\int_1^2 x^2 w' y' \, dx + \int_1^2 x w y' \, dx + \int_1^2 (\pi^2 + 1) w y \, dx = 0.
\]

\subsection*{Elemental Contribution}

For each element $[x_j, x_{j+1}]$:
\[
\int_{x_j}^{x_{j+1}} x^2 \varphi_i' \varphi_j' \, dx + \int_{x_j}^{x_{j+1}} x \varphi_i \varphi_j' \, dx + \int_{x_j}^{x_{j+1}} (\pi^2 + 1) \varphi_i \varphi_j \, dx.
\]

Element stiffness matrix $K_{ij}$ and load vector $F_i$ formed by these integrals using Galerkin method.

\subsection*{Exact Solution}

\[
y(x) = 62.5 \, x \sin(2.5 \ln x).
\]

\subsection*{MATLAB Code to Compare Solutions}

\lstset{
    language=Matlab,
    basicstyle=\ttfamily\footnotesize,
    keywordstyle=\color{blue},
    commentstyle=\color{gray},
    stringstyle=\color{red},
    numbers=left,
    numberstyle=\tiny\color{gray},
    stepnumber=1,
    numbersep=5pt,
    frame=single,
    breaklines=true,
    showstringspaces=false
}

\begin{tcolorbox}[title=Finite Element Method (Piecewise Linear Basis)]
\begin{lstlisting}
clc; clear;
N = 50;
h = 1 / N;
x = linspace(1, 2, N+1);
y1 = 0;
y2 = 2 * sin(pi * log(2));
a = zeros(1, N-1);
b = zeros(1, N-1);
c = zeros(1, N-1);
for i = 1:N-1
    xi = x(i);
    a(i) = 1/h^2 + xi / ((1 + xi^2) * h);
    b(i) = -2 / h^2;
    c(i) = 1/h^2 - xi / ((1 + xi^2) * h);
end
d = -1 ./ (1 + x.^2);
f = zeros(1, N-1);
f(1) = d(2) + y1 * ( -x(1) / ((1 + x(1)^2) * h) - 1/h^2 );
for i = 2:N-2
    f(i) = d(i+1);
end
f(N-1) = d(N) + y2 * ( x(N-1) / ((1 + x(N-1)^2) * h) - 1/h^2 );
b1 = b;
f1 = f;
for i = 2:N-1
    m = a(i) / b1(i-1);
    b1(i) = b1(i) - c(i-1) * m;
    f1(i) = f1(i) - f1(i-1) * m;
end
u = zeros(1, N-1);
u(N-1) = f1(N-1) / b1(N-1);
for i = N-2:-1:1
    u(i) = (f1(i) - c(i) * u(i+1)) / b1(i);
end
y_approx = [y1, u, y2];
y_exact = x .* sin(pi * log(x));
error = abs(y_approx - y_exact);
figure;
plot(x, y_approx, '-.b', x, y_exact, 'r', x, error, '--k', 'LineWidth', 2);
title('BVP Exact and Finite Element Solution', 'FontSize', 16);
xlabel('x', 'FontSize', 14);
ylabel('y', 'FontSize', 14);
legend('FEM Approximation', 'Exact Solution', 'Error');
grid on;
\end{lstlisting}
\end{tcolorbox}

\begin{figure}[h!]
    \centering
    \includegraphics[width=0.8\textwidth]{image3.jpg}
    \caption{BVP Exact and Finit Elemnt Solution for $u'' + u' = 5$}
    \label{fig:fem_solution}
\end{figure}

\end{document}
